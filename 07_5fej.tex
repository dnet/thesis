\chapter{Results}

\section{Measurement}

\subsection{Evironment}

% localhost: hw, network
% same server
% versions: java, cxf, python, suds
% Wireshark
% LibreOffice

\subsection{Methodology}

% code instrumentation
% time vs clock
% log file
% CSV file
% config in JSON

\section{Analysis}

\subsection{Proxy initialization}

\begin{figure}[htbp]
 \begin{center}
  \input{stat_init}
  \caption{Time needed for CXF and SUDS proxy initialization}
  \label{fig:stat_init}
 \end{center}
\end{figure}

\noindent
The first thing that catches the eye on Figure \ref{fig:stat_init} is that SUDS instantiates the proxy in less then half the time CXF needs to perform the same, regardless the configuration. It's also interesting to see, that according to these timings, SUDS needs an additional \mbox{210 ms} on average in case of digital signatures, whereas CXF takes roughly the same time in every case. The most logical explanation to me is that in the SUDS client, library import is done on demand, so this extra time is what it takes to load the LXML, libxml2 and XMLSec libraries, latter two requiring native components.

\subsection{Invocation round-trip time}

\begin{figure}[htbp]
 \begin{center}
  \input{stat_invoke}
  \caption{Time needed for CXF and SUDS invocation}
  \label{fig:stat_invoke}
 \end{center}
\end{figure}

\subsection{Network traffic}

% TCP keep-alive
% headers
% packet count

\section{Architectural differences}

\subsection{Runtime}

% VM startup time & memory consumption

\subsection{Use of native components}

% performance vs crashing

\chapter{Summary}

\section{Results summary}

\section{Future development opportunities}

Since most of the solutions in the Python ecosystem are free software in both senses\footnote{free as in free beer vs. free as in free speech}, software is continuously improved by its community following the needs of their members. For instance, during my thesis, I also created and published a proof-of-concept code to handle messages with attacments, and an interested user developed it further since to the point of real world usability. The following three goals are those, I think can and should be done in order to further improve Python SOA solutions.

\subsection{Short-term: taking advantage of HTTP keep-alive}

The current implementation of SUDS uses \emph{urllib2}, part of the Python base library, as I wrote in section \ref{sudsInvocation}. This solution opens a new TCP connection for each HTTP request by default, which causes overhead, especially a problem in case of HTTPS, which requires a TLS handshake in addition to the TCP one. Several ways exist to solve this issue -- as explained in \cite{so-1037406} -- some require replacing \emph{urllib2} with a whole new library, others just require extending it with plugins (so-called \emph{openers}) -- two things are for sure; it requires more than a simple method call, and the solution must be carefully tested for compatibility with different versions of its dependencies. Still, it only affects a relatively isolated part of the codebase, and would improve performance regardless of any optional (security) features.

\subsection{Mid-term: implementing XML encryption}

As I mentioned in section \ref{xmlenc}, as of 2011, the XML encryption standard is considered broken, so implementing it would do little or no good -- in my opinion, false belief in security is worse than no security at all. But as the future goes, XMLSec supports XML encryption, so in case the library is updated against the new design, it'd be fairly straightforward to make use of it in SUDS and sec-wall. I consider it a mid-term goal, since in this case (in contrast with digital signature), it makes more sense to use encryption on the response too, so the solution should be usable in both scenarios. By looking at Figure \ref{fig:sudsMessage}, it's clear, that the \emph{received} hook could be used to construct a message plugin the same way I did implementing the \emph{SudsSigner}.

\subsection{Long-term: wider cryptographic backend support}

The current \emph{SudsSigner} implementation supports only DSA and RSA digital signatures, and while these are quite common because of the ubiquity of PKI implemented with X.509 certificates, the WS\hyp{}Security standard -- or more specifically, the XML Signature W3C Recommendation -- contains guidelines for OpenPGP, too. As the main developer, Aleksey Sanin wrote in \cite{aleksey-pgp-mail}, XMLSec could have had PGP support, but he had problems regardning the availability and licensing of the necessary library. Two years later, John Belmonte wrote in \cite{belmonte-pgp-mail}, that he took part in a project that implemented PGP XML digital signatures in Python, although it seems by his words, that the product is closed source. That said, it doesn't seem impossible to add PGP support into XMLSec, using an appropriate library -- and GPGME \cite{gpgme-homepage} seems like the best candidate.
