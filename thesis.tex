%diploma-vaz fo dokumentum
%Keszult: 2004. februar
%(c) Markus Janos, http://mit.bme.hu/~markus, markus@mit.bme.hu
%Tesztelve: Miktex 2.3 alatt

%\documentclass[12pt,a4paper,twoside,openright]{report}  %Ketoldalas szedes
\documentclass[12pt,a4paper,oneside]{report}            %Egyoldalas szedes

%%Itt kivalaszthatjuk az egyes fejezeteket, ha nem akarjuk az egeszet forditani
%\includeonly{01_elozm, 02_eloszo, 03_1fej}

\usepackage{bmedipl}         %margo-, nyelv es egyeb allitas

%\usepackage{amsmath}         %matematikai segelycsomag
\usepackage{graphicx}        %grafikai
\usepackage{amssymb}
\usepackage{listings}
\usepackage{booktabs}
\usepackage{ltablex}
\usepackage{tikz}
\lstset{numbers=left, numberstyle=\tiny, basicstyle=\ttfamily, breaklines=true, frame=single, tabsize=2}

\usepackage{setspace}         %1.5-os, 2-es sorkoz hasznalatahoz.
                              %Ettol a tablazatok, abrak, labjegyzetek maradnak 1-es sorkozzel!
\onehalfspacing               %1.5-os sortav. Nem kotelezo szerintem...
%\doublespacing                %2-es sortav. Csak korrekturahoz!...

%\usepackage[dcu]{harvard}    %harvard tipusu hivatkozashoz (ld. a dokumentaciojat)
% \citationstyle{dcu}
% \citationmode{abbr}
% \harvardparenthesis{square}
% \harvardyearparenthesis{round}
% \renewcommand{\harvardand}{\'es}


%a jelolt neve
\renewcommand{\nev}{András Veres-Szentkirályi}

%konzulens adatai
\renewcommand{\konzulens}{Balázs Simon}
\renewcommand{\konzbeoszt}{PhD student}

%a dolgozat cime, ev
\renewcommand{\cim}{Extending Python Web Services}
\renewcommand{\ev}{2011.}

\hyphenation{meg-szentség-te-le-nít-he-tet-len}  %egyedi elvalasztas

\usepackage[
  unicode=true,
  colorlinks=false,
  pdfborder={0 0 0 0},
  pdfauthor={\nev},
  pdftitle={\cim}
]{hyperref}
\usepackage[all]{hypcap}

\begin{document}

\pagenumbering{roman}
%%%%%%%%%%%%%%%%%%%%%%%%%%%
% Diplomaterv-kiiras (ezt adjak, bele kell kötni a diplomába)
%%%%%%%%%%%%%%%%%%%%%%%%%%%
\begin{center}

\textbf{BUDAPEST UNIVERSITY OF TECHNOLOGY AND ECONOMICS}

\medskip

\includegraphics[width=8.79cm]{images/bme.pdf}

\medskip

\textbf{FACULTY OF ELECTRICAL ENGINEERING AND INFORMATICS\\SOFTWARE ENGINEERING}

 \vspace{2cm}
 \Large\textbf{\cim}

 \vspace{6mm}
 \textbf{\nev} \\
 \texttt{<vsza@vsza.hu>} \\ \strut \\

 \Large\textbf{THESIS STUDY}

\end{center}

\vfill

Consultant:

\begin{center}
\konzulens\\ \konzbeoszt

\vspace{96pt}

December 2011
\end{center}

% \vspace{6mm}
% \begin{tabular}{p{80mm}l}
% A záróvizsga tárgyai:   & Első tárgy \\
%                         & Második tárgy \\
%                         & Harmadik tárgy
%  \end{tabular}
%
%  \vspace{6mm}
%  \begin{tabular}{p{80mm}l}
%  A tervfeladat kiadásának napja:         &  \\
%  A tervfeladat beadásának határideje:    &
%  \end{tabular}
%
% \vfill
%
% \begin{center}
% \begin{tabular}{cc}
%  \makebox[7cm]{\emph{dr.\ Görgényi András}}    & \makebox[7cm]{\emph{dr.\ Péceli Gábor}} \\
%  \makebox[7cm]{adjunktus, diplomaterv felelős} & \makebox[7cm]{egyetemi tanár, tanszékvezető}
% \end{tabular}
% \end{center}
%
%  \vspace{6mm}
%  \begin{tabular}{p{80mm}l}
%  A tervet bevette:           & \\
%  A terv beadásának dátuma:   & \\
%  A terv bírálója:            &
%  \end{tabular}


 \thispagestyle{empty}
 \blankpage

\selectlanguage{magyar}
%%%%%%%%%%%%%%%%%%%%%%%%%%%
% Diplomaterv-kiiras melleklete (ezt is adjak, bele kell kötni a diplomába)
%%%%%%%%%%%%%%%%%%%%%%%%%%%
 \begin{textblock*}{\paperwidth}(0mm,0mm)
    \noindent\includegraphics[width=\paperwidth,height=\paperheight]{images/feladat-retouched.jpg}
	\end{textblock*}
	\mbox{}
 \blankpage

%%%%%%%%%%%%%%%%%%%%%%%%%%%
% Nyilatkozat
%%%%%%%%%%%%%%%%%%%%%%%%%%%
\def\abstractname{Nyilatkozat}
\begin{abstract}

\noindent
Alulírott \emph{Veres-Szentkirályi András}, szigorló hallgató kijelentem,
hogy ezt a diplomatervet meg nem engedett segítség nélkül, saját  magam
készítettem, csak a megadott forrásokat (szakirodalom, eszközök, stb.)
használtam fel. Minden olyan  részt, amelyet szó szerint, vagy azonos
értelemben, de átfogalmazva más forrásból átvettem, egyértelműen, a
forrás megadásával megjelöltem.

Hozzájárulok, hogy a jelen munkám alapadatait (szerző(k), cím, angol és magyar
nyelvű tartalmi kivonat, készítés éve, konzulens(ek) neve) a BME VIK nyilvánosan
hozzáférhető elektronikus formában, a munka teljes szövegét pedig az egyetem
belső hálózatán keresztül (vagy autentikált felhasználók számára) közzétegye.
Kijelentem, hogy a benyújtott munka és annak elektronikus verziója megegyezik.
Dékáni engedéllyel titkosított diplomatervek esetén a dolgozat szövege csak 3 év
eltelte után válik hozzáférhetővé.
\begin{flushright}
 \vspace*{1cm}
 \makebox[7cm]{\rule{6cm}{.4pt}}\\
 \makebox[7cm]{\emph{Veres-Szentkirályi András}}\\
 \makebox[7cm]{hallgató}
\end{flushright}
\end{abstract}

%%%%%%%%%%%%%%%%%%%%%%%%%%%
% Tartalomjegyzek
%%%%%%%%%%%%%%%%%%%%%%%%%%%
\selectlanguage{english}
\tableofcontents

%%%%%%%%%%%%%%%%%%%%%%%%%%%
% Kivonat
%%%%%%%%%%%%%%%%%%%%%%%%%%%
\def\abstractname{Kivonat}
\selectlanguage{magyar}
\begin{abstract}
\addcontentsline{toc}{chapter}{Kivonat}
% TODO
\end{abstract}


%%%%%%%%%%%%%%%%%%%%%%%%%%%
% Abstract
%%%%%%%%%%%%%%%%%%%%%%%%%%%
\selectlanguage{english}
\begin{abstract}
\addcontentsline{toc}{chapter}{Abstract}
``If I have seen further it is only by standing on the shoulders of giants'' -- the more than 300 years old message of Isaac Newton is a great parallel with the motivation behind service interoperation. As more advanced services were developed, one way of improving them was to interconnect them to combine their powers into more exciting products. Continuous advancement of technology created diverging platforms, and simultaneously provided standards allowing for efficient co-operation such as DCOM, CORBA and XML-RPC.

One of the solutions for this Babel-like chaos were web services, using SOAP to encode business messages in a standardized way, and reusing simple and mature transport mechanisms proven useful by the World Wide Web to carry them between the recipients. While the openness Internet offers endless opportunities, it also has its dangers, which caused additional standards to be developed, among others, for message authencity.

Although the open standards of SOAP, WSDL and others could have been the foundation of a platform-independent solution, not every environment used for software development supports it equally. Python, a truely community-driven project is one of them, providing little more than minimal support for advanced SOAP web services, making it a less favored selection for projects needing this capability, despite its unique treats.

In this thesis, the history and principles of Service-oriented Architecture are presented, then the scope is focused on web services, and further on to advanced ones. Then the Python environment is introduced, including the current solutions for implementing SOA solutions both on the service and consumer side. This part ends with a quick summary, that makes the reasons for improvements clear.

My selection for improvement, SUDS is presented next, looking at both its high-level view and its internals, showing the possible stubs awaiting improvement. The plans and implementation details of my enhancements are introduced right after, including a testbed to make sure, the new features fulfill the most important requirement: interoperation. In the end, the whole solution is evaluated using measurements of both timing and network traffic, concluding the thesis with my observations and ideas for future improvement.
\end{abstract}
      %elso lapok (Cimlap, kiiras, tartalomjegyzek, Kivonat, Abstract,egyeb)
%%Eloszo

\chapter*{Introduction}

 \addcontentsline{toc}{chapter}{Introduction}
 \markboth{\uppercase{Introduction}}{\uppercase{Introduction}}
 \pagenumbering{arabic}

% background in five paragraphs
Looking at the history of ITC systems, interoperability was not a big issue at the beginning -- simple systems can communicate using primitive methods. As time went by, systems evolved, and innovation lead to such diversity, that made high level interconnection of systems a major headache of system integrators. Naturally, the market reacted and came up with various solutions, well distributed along the scale of bloatedness, including DCOM, CORBA and XML-RPC.

These competing solutions were and are more or less usable within their scopes, limited by platform-dependence, but more importantly the inability to operate over the Internet -- either because of the incompatibility with appliances at network borders, or because of security issues. The World Wide Web brought simple open protocols, and mature solutions to pass its network traffic through network borders, making it an ideal choice as the transport layer for the next generation of interoperability platforms.

SOAP was created as the encoding method of request, reply and fault messages exchanged between services and consumers over the transport layer, but it didn't solve all problems in itself. For instance, trusting a network out of control of both parties required additional standardized ways of ensuring the confidentiality and integrity of the transmitted message, as well as authenticating the consumer and/or the service. In case of this problem, WS\hyp{}Security was born as a solution, providing a simple and open method of ensuring the necessary level of security for SOAP messages relayed over untrusted networks.

Python was one of the few languages, that -- despite its roots -- could emerge from the academic circles, and found its way to developers, hackers, and system administrators alike. The rich set of libraries and sane design made it a perfect choice for high-level implementation, allowing a smooth transition from ideas through prototypes to solutions ready for deployment -- the same feature that caused many people writing it off as a ``scripting language''. As was Java before the millennium, Python is currently considered by many as the language of the Internet (or at least one of them), which means, advanced SOAP support is a must-have in order for Python to be accepted as the foundation of a wider range of systems.

One of the negative side effects of the community-driven development of the Python ecosystem is that features needed by less people get a smaller fraction of developer attention -- and this was exactly the fate of advanced Python SOA implementations. Although SOAP libraries did exist, their support was limited to the level the developer(s) required, resulting in many organizations choosing other platforms solely based on their advanced SOAP support -- creating a situation I refused to accept.

% sum of chapters in two paragraphs
The first chapter introduces the Service-oriented Architecture and web services, including their brief history and principles. Using these as a basis, the scope is first focused on advanced web services, and then on the security of web services. The second chapter sheds some light on the other half of the thesis title, Python, and presents the available SOA solutions compatible with the platform, along with their advantages and disadvantages. The chapter ends with a quick summary of the Python SOA landscape, picking SUDS as a suitable base of improvement.

The third chapter goes into detail about SUDS, starting with its interface presented to the developer, then focusing on the security-related parts of its internals, ending in its interestring stubs awaiting improvement. The fourth chapter starts from the deficiencies of SUDS, and introduces the component I designed to enable advanced web service consumptions. In the second half of the chapter, a testbed is shown, which I developed to make sure, that the result of the improvement is able to interoperate with services correctly. The fifth chapter demonstrates the measurements I did to evaluate the quality of my solution, and the sixth closes the thesis study by summarizing the development process and its results, offering future improvement ideas to address the remaining issues.
     %Eloszo, ebben celkituzes, elozmenyek, felepites, koszonetnyilvanitas
                        %max. nehany oldal

%\chapter{Service-oriented Architecture and Web Services}

\section{SOA history and principles}

As \cite{soa_modeling} remembers, not long after the new millenium, the world of IT got fed up with interoperability, reusability, and other issues -- and Service-oriented Architecture was born. The paradigm was was built upon the foundations of IT best practices of its time, and tried to encourage software design made of loosely coupled components. Reduction of time to market and business agility are advantages, that are obvious to both business and IT people. This step is a logical one in the course of software engineering history -- states \cite{devcom_soa_intro}. The technological shifts always followed the increasing software complexity, from functions, through classes, to components. But even the users of components are tied to the technology (runtime, platform) the component uses. According to \cite{ibm_soa_impro}, SOA addresses this problem by the following guiding principles.

\begin{itemize}
 \item Reuse, granularity, modularity, composability, and componentization
 \item Compliance to standards (both common and industry-specific)
 \item Services identification and categorization, provisioning and delivery, and monitoring and tracking
\end{itemize}

\section{Web Services}

Many technologies tried to implement SOA (or something close), for example Microsoft's DCOM and OMG's CORBA also offered a somewhat standardized way for entities (components, services) to interoperate. One of the problems were the limitations of the implementations -- DCOM obviously depended on Windows as a platform, and although CORBA had (and has) ORB implementations available to several platforms and under diverse licensing, but the interoperability between these was often an issue. An even more troubling problem was the communications foundation of these solutions. Most of these (DCOM and CORBA at least for sure) used a binary protocol and required direct connections to TCP ports -- sufficient for components communicating within the local corporate network, but unimaginable over the internet.

The World Wide Web introduced HTTP as a transport protocol, one especially designed for use over the internet. Besides that, corporate networks could also make use of it, since proxy servers could be installed, with the option of inspection, forwarding, filtering, and mangling of content passing through. These two properties made it a great foundation for interoperation, since the protocol allows any kind of content to be transferred, regardless of its type.

In a wide sense, every service, that is available for invocation through HTTP can be considered a web service, regardless of the layer used above HTTP. XML-RPC was the first such ``payload'', and as the name suggests, its semantics were based on method invocation. The structure is suprisingly simple, the method name and parameters are transmitted as the body of an HTTP request, and the body of the response contains the return value(s), both serialized using XML. A sample transcript can be seen on Figure \ref{fig:xmlrpc-sample}.

\begin{figure}[htbp]
 \centering
 \begin{minipage}[t]{0.56\linewidth}
  \centering
  \begin{lstlisting}[language=XML, numbers=off]
<?xml version="1.0"?>
<methodCall>
 <methodName>getPop</methodName>
 <params>
  <param>
   <value>
    <string>Budapest</string>
   </value>
  </param>
 </params>
</methodCall>
  \end{lstlisting}
 \end{minipage}
 \hspace{0.5cm}
 \begin{minipage}[t]{0.37\linewidth}
  \centering
  \begin{lstlisting}[language=XML, numbers=off]
<?xml version="1.0"?>
<methodResponse>
 <params>
  <param>
   <value>
    <i4>1733685</i4>
   </value>
  </param>
 </params>
</methodResponse>
  \end{lstlisting}
 \end{minipage}
 \caption{Transcript of an XML-RPC method invocation}
 \label{fig:xmlrpc-sample}
\end{figure}

\section{SOAP and friends}

\begin{figure}[htbp]
 \centering
 \begin{minipage}[t]{0.47\linewidth}
  \centering
  \begin{lstlisting}[language=XML, numbers=off]
<soap:Envelope xmlns:soap="http://schemas.xmlsoap.org/soap/envelope/">
 <soap:Body>
  <bme:getPop xmlns:bme="http://vsza.hu/bme">
   <city>Budapest</city>
  </bme:getPop>
 </soap:Body>
</soap:Envelope>
  \end{lstlisting}
 \end{minipage}
 \hspace{0.5cm}
 \begin{minipage}[t]{0.47\linewidth}
  \centering
  \begin{lstlisting}[language=XML, numbers=off]
<soap:Envelope xmlns:soap="http://schemas.xmlsoap.org/soap/envelope/">
 <soap:Body>
  <bme:getPopResponse xmlns:bme="http://vsza.hu/bme">
   <return>1733685</return>
  </bme:getPopResponse>
 </soap:Body>
</soap:Envelope>
  \end{lstlisting}
 \end{minipage}
 \caption{Transcript of a SOAP method invocation}
 \label{fig:soap-sample}
\end{figure}

As \cite{box_soap_history} wrote, SOAP has evolved from XML-RPC inside Microsoft -- the base operation remained the same, the method identification and parameters are serialized using XML, and so is the response. One notable difference is the absence of XML header (SOAP uses UTF-8 encoding implicitly) and the extensive use of XML namespaces, as it can be seen on Figure \ref{fig:soap-sample}. W3C took control of the specification, and SOAP became the encoding of web services, with usually HTTP(S) or SMTP as the underlying transport mechanism. WSDL was born to describe the interface of web services, using XML again. Although other technologies appeared (such as UDDI for service discovery) the two dominant players in web services are SOAP and WSDL.

WSDL is usually automatically generated from services written in any programming language. Client libraries running on platforms supporting dynamic dispatch (such as Python, Ruby, PHP) usually allow dynamic creation of service proxies for consumption using the WSDL. The other approach, available for all runtimes and most programming languages is automatized code generation, during which class hierarchies representing the service interfaces are generated for later used in compiled code. During invocation, the proxy serializes the platform-dependent data structures into a SOAP envelope, and transfers it to the service, which does the exact opposite by marshalling the parameters into native objects. This way, both the service and the consumer code handles entities that are native to the platform they're dependent on, and can interoperate with each other, without any prior knowledge of the technology powering the ``other side''.

\section{Advanced web services}

% TODO
       %elso fejezet: bevezetes
%\chapter{Existing Python SOA solutions}

\section{About Python}

\subsection{Language}

According to \cite{python-faq}, ``Python is an interpreted, interactive, object-oriented programming language''. What's missing from this self-description are those things, that make the language unique. The first thing most people recognize while reading a Python source code, is the use of indentation for structure markup. This feature might be odd and strict for first sight, but it makes code written by other people highly readable and reusable. The subsequent clean feeling of the code is strengthened even more by the availability of functional constructs, which give the right tools for most purposes into the hands of the developer. The essence of the language can be reduced to a single phrase, which can be interpreted in both positive and negative ways: ``executable pseudocode'' -- code snippets are explicit enough for most people (even those without Python knowledge) to understand.

\subsection{Runtime}

The first and most widely-used interpreter is called CPython, and it's the reference implementation of the language runtime. It compiles source code (\verb|.py| files) into bytecode (\verb|.pyc| files) for interpretation. It also provides an interactive shell, which can be used for experimentation or debug purposes. Because of this, the UNIX program loader can use it as a standard interpreter, so Python scripts prefixed with an appropriate shebang can be run directly. As the name suggests, the implementation is written in mostly C/C++, which causes built-in functions to perform well.

There are separate projects, that bridges the Python world with other solutions -- IronPython and Jython compiles Python code into .NET and Java bytecode, respectively, and Nokia ported Python to its S60 (Symbian) platform. This way, Python can interoperate with existing frameworks and libraries at a lower level, if needed. Another approach is outlined in the next subsection.

\subsection{Libraries}

Python comes with ``batteries included'' -- libraries are available for most purposes a developer might need, such as file manipulation, network connectivity, parsing and serializing from and to a variety of formats. Libraries can be either written in Python -- in which case, they are as portable as any other Python code between runtimes -- or using the C/C++ API. Thin, sometimes automatically generated libraries, that only wrap a certain native libraries are called bindings -- there's even a dialect of Python called Cython, that allows calling of C/C++ functions, and produces native code. Because of these features, although interpreted languages are usually suffer from poor performance, well-designed Python applications perform only high-level orchestration in the interpreted engine, and delegate computationally intensive tasks to native code. This design motivates developers to avoid premature optimization, while allowing fast prototyping and outstanding performance using the same foundations.

\section{Python SOA solutions}

\subsection{Introduction}

The community around the Python ecosystem is one of those closest to the ``free software culture'' envisioned by Richard M. Stallman and Eric S. Raymond. This results in libraries being written mostly out of curiosity and immediate need -- a good combination for a good base system, not so good for SOAP. Many other ways of remote method invocation are solved in Python libraries, but SOAP has maintained a low level of maturity. The basic invocation examples usually work, but the level of development clearly shows the needs of the developer.

This problem is partly caused by the network effect: if everybody uses .NET and Java for web service interoperation, if a platform needs to be chosen for a solution to access them, it usually seems logical for most people to choose one of the two heavyweight products. The other factor is the mindset of Python developers -- they usually like to build systems out of small, autonomous entities, interconnected by simple and trivial protocols, and SOAP is not the first thing that comes to mind with these features, despite the fact that it can be used wisely.

Of course, there are developers both working on and using SOA with Python, there's even a public mailing list dedicated for the purpose, archives and subscription are available at \url{http://mail.python.org/mailman/listinfo/soap}.

\subsection{SOAPy}

As \cite{so-206154} wrote, it was the best SOAP client in the Python ecosystem, but the project is abandoned. As no one maintains the codebase -- its homepage was last modified in 2001 -- it became incompatible with later Python releases, which makes it hard to use in modern environments. The Debian project doesn't even maintain a package, so the only way to install it is to download the tar.gz file (uploaded in April 27, 2001) and extract it manually.

I found its internal structure very simple -- the library consists of two Python source files, both under 500 lines of length. By looking at the import section, it was obvious, that it used libraries and functions, that are way obsolete now. Still, some of them are kept for the sake of backwards compatibility, but the PyXML package it used for XML processing is also no longer maintained, and had been removed from most major GNU/Linux distributions. The documentation -- including the examples -- suggested, that SOAPy offered client functionality only, and I didn't find any contradicting evidence in the source code.

\subsection{Zolera SOAP Infrastructure}
\label{ZSI}

ZSI is the other ``old boy'' among Python SOAP libraries. As \cite{pywebsvcs-talk} states, it way last fully released in 2007, but unlike SOAPy, it can be still used with recent Python environments. It offers two ways of operation: for simple services, it can construct the SOAP messages without a schema (Binding class), and for complex services, a proxy (ServiceProxy class) can be used to serialize arguments. It supports code generation from WSDL (\verb|wsdl2py|) and ``can also be used to build applications using SOAP Messages with Attachments'' \cite{zsi-doc}.

Ralf Schmitt wrote in \cite{zsi-velocity}, that ZSI is neither easy to set up and use, nor fast. I tried it anyway, and it was easy to install, since Debian still maintains a package. The first ServiceProxy example, which I took straight from the ZSI documentation failed, but I figured out, that they changed the structure, so I managed to run a test. It worked pretty well as a client, but it lacked any advanced debugging features -- for example, the list of methods could only be determined by listing the methods of the service proxy. Also, code (re)generation is necessary for complex data types, and is far from being trivial.

\subsection{soaplib / rpclib}

Soaplib focused on server-side SOAP implementation, using Python decorators, and provided WSGI-compatible services, so deployment was possible with both standalone processes or any WSGI-compatible web server (such as Apache \verb|mod_wsgi|). \cite{so-206154} wrote, that ``creating clients is a little bit more challenging'', so I looked through the documentation, and found, that according to \cite{soaplib2-changelog}, the developers did ``the right thing'' and shifted their entire focus on server implementation by dropping client functionality in favor of SUDS (see section \ref{suds}) at version 0.9.

The project was later renamed to rpclib, and widened its scope -- the old library only receives bug fixes, as the main developer focuses on the new one. I tried using it, and found it pleasant to use. Despite Python being a dynamically typed language, enabling code to be accessible via SOAP had not ``littered'' the code, and it offered automatic WSDL generation.

\subsection{SUDS}
\label{suds}

SUDS is a relatively new SOAP client library, compatible with Python 2.4 and newer releases. Its operation is like the proxy feature of ZSI, but it doesn't require any code generation. Complex classes can be assembled using the factory pattern, and while it might seem, that parsing WSDL and generating class hierarchy on-the-fly is slow, the built-in caching provides quite a performance. It supported several methods of authentication, including HTTP basic and digest, and also NTLM, which is necessary to consume Microsoft SharePoint web services. According to the general opinion of related forums and mailing lists (including \cite{so-206154}), SUDS is the preferred Python way of creating SOAP clients, and the library doesn't depend on obsolete components.

SUDS was released in a regular manner till 2010, and is available as a package in major Linux distributions. This way, installing the library was not a big issue, and the documentation \cite{suds-doc} covers all common use-cases. I tried it first with a basic invocation, and it worked as expected. Special methods were overridden in a way, that using the \verb|print| command on SUDS object rendered a nicely formatted, human readable printout, which makes debugging and experimentation much easier.

\subsection{sec-wall}

Although \cite{sec-wall-homepage} describing sec-wall as ``a security proxy that comes with tons of interesting features, very good documentation and an exceptionally friendly community'' might sound like the usual shameless self-promotion, this relatively new tool (1.0 released in April 2011) is a real gem. It acts as a proxy, thus enables the transformation of any SOAP backend into an advanced web service. Although written in Python could have meant poor performance, it takes the issue seriously and makes use of libraries that provide an native event-driven architecture.

I found it during the field-work, and I worked together with its author, Dariusz Suchojad to improve it -- one of the results were complete and correct UsernameToken support (both plain and digest), and an experimental digital signature implementation. Beside the performance and reusability, the quality of the software is also surprisingly great; it's built around the Python Spring Framework, making good use of the dependency injection feature, and its tests provide 100\% code coverage.

\subsection{Common problems}

While inspecting libraries offering both service and consumer functionality, unfortunately, they provided no or little support for advanced web services. SUDS offered UsernameToken, but didn't work in any mode, soaplib/rpclib and ZSI didn't even mention the possibility of such solutions -- although ZSI had some unused code implementing XML canonicalization. SOAPy barely even implemented SOAP -- besides it's unusable in modern environments. Although sec-wall solves the situation by providing proxy support, the problem of the client side remains -- and that's exactly, why I decided, to take a close look into SUDS.
       %masodik fejezet: a problema reszletes kifejtese
%\chapter{Opportunities and internals of SUDS}

\section{Introduction}

As I described in section \ref{suds}, SUDS is the de facto way of consuming web services in Python. One of the most compelling features lies within its simplicity and user friendliness. These help in the beginning, by making it really easy to create a working prototype in no time, both by using the interactive shell and writing scripts -- but later, the code is still readable, and at the same time, caching helps eliminating the performance trade-off. A sample run, consuming a currency rate service using SUDS in the interactive Python shell can be seen in Figure \ref{fig:suds-currency}.

\begin{figure}[htbp]
 \centering
\begin{lstlisting}[numbers=off, basicstyle=\footnotesize\ttfamily]
Python 2.7.2+ (default, Aug 16 2011, 07:03:08)
[GCC 4.6.1] on linux2
Type "help", "copyright", "credits" or "license" for more information.
>>> from suds.client import Client
>>> url = 'http://www.webservicex.net/CurrencyConvertor.asmx?WSDL'
>>> c = Client(url)
>>> print c

Suds ( https://fedorahosted.org/suds/ )  version: 0.4.1 (beta)  build: R703-20101015

Service ( CurrencyConvertor ) tns="http://www.webserviceX.NET/"
   Prefixes (1)
      ns0 = "http://www.webserviceX.NET/"
   Ports (2):
      (CurrencyConvertorSoap)
         Methods (1):
            ConversionRate(Currency FromCurrency, Currency ToCurrency, )
         Types (1):
            Currency
      (CurrencyConvertorSoap12)
         Methods (1):
            ConversionRate(Currency FromCurrency, Currency ToCurrency, )
         Types (1):
            Currency


>>> c.service.ConversionRate('EUR', 'HUF')
315.6003
\end{lstlisting}
 \caption{Requesting currency conversion rate using SUDS}
 \label{fig:suds-currency}
\end{figure}

\section{Internal structure}

In order to improve SUDS, I had to discover its inner workings -- the documentation covered standard use-cases pretty well, but told little about architecture. I split the code in time domain into two pieces, the separator being the end of \emph{suds.client.Client} object instatiation. Before that, WSDL fetching and parsing happens, and afterwards, during invocations, SOAP messages are built, sent, and responses are parsed and returned.

\subsection{Client proxy instantiation}

% TODO

\subsection{Service method invocation}

% TODO

\subsection{Document Object Model of SUDS}

As \cite{w3schools-domintro} defines it, ``XML DOM is a standard for how to get, change, add, or delete XML elements'', which is the better way to construct XML output -- the worse being string concatenation. SUDS has its own implementation, and as \cite{suds-doc} states, it ``was written [because] elementtree and other [Python] XML packages either: have a DOM API which is very unfriendly or: (in the case of elementtree) do not deal with namespaces and especially prefixes sufficiently'' -- and in retrospect, it was a perfectly sane decision back then. The SUDS DOM resides in the \emph{suds.sax} module, and interfaces the outside world with the Python built-in SAX parser. It registers itself as a SAX event handler, and builds the document tree from its own objects in response to parsing events, so there is a clear separation between the Python XML library and the implementation of SUDS.

Although now we have LXML (see section \ref{lxml}) which would have satisfied those conditions (and is used by rpclib), it was probably not in this state of maturity, when the SUDS project kicked off. It has its own peculiarities, such as namespace handling is done using (prefix, namespace) tuples -- in constrast with standard notations such as dictionary objects or James Clark style. This self-developed solution also caused the appearance of ``double namespaces'' -- the SOAP-ENV namespace was declared with one prefix for the envelope and header, and another for the body. While working on improving SUDS, I also found that it had several deficiencies, for instance, there's no way of handling attributes with namespaces. It could seem, that now it'd be time to replace the library with a thin wrapper around LXML or some other functionally equivalent components, but it'd break existing code depending on the internals of SUDS.

\section{Opportunities}

\subsection{Current WS-Security implementation}

\subsubsection{Timestamp}

% TODO

\subsubsection{UsernameToken}
\label{sudsUsernameToken}

% TODO

\subsection{Plugin system}

% TODO
       %harmadik fejezet: eddig alkalmazott megoldas

\chapter{Improving SUDS}

\section{Implementing digital signatures -- SudsSigner}

\subsection{Internal structure}

The internal structure of the plugin can be seen on Figure \ref{fig:cdSudsSigner}, the stereotypes describe the functionality (component or binding) and the runtime environment (Python or native) of each component. Native components are preferred for their reusability and performance -- reusable components are tested more thoroughly, as more projects can depend on them, which makes them less prone to errors, thus more suitable for security-critical tasks, such as cryptography (OpenSSL). From a performance point of view, XML parsing and processing is also a task, that is better done using native and mature code (libxml2) because of its complexity. Python, on the other hand, is more suitable for the purpose of connecting components together, and describe high-level business logic in a readable and portable way.

\begin{figure}[htbp]
 \centering
 \includegraphics[width=\textwidth]{images/cdSudsSigner.pdf}
 \caption{Component diagram of the SudsSigner plugin}
 \label{fig:cdSudsSigner}
\end{figure}

\subsection{Components used}

\subsubsection{Libxml2, python-libxml2 and LXML}

``Libxml2 is the XML C parser and toolkit developed for the Gnome project (but usable outside of the Gnome platform), it is free software available under the MIT License.''\cite{libxml2-homepage} This sentence summarizes the project pretty well -- it's written in C, which is a good compromise between performance, portability and usability, and it's available under the MIT license, which makes it possible to either bundle it to FLOSS projects or redistribute it with proprietary software. Since many projects depend on it, the quality of the code is high, and it passes all of the OASIS XML Tests Suite.

Python-libxml2 provides low-level Python bindings to access all of the functionality of libxml2. It has the advantages that the developer gets the full power of libxml2, but the interface resembles the original C API, which causes longer development and debug cycles. On the other hand, LXML wraps libxml2 in modules and classes providing a powerful high-level interface, which is more suitable for quick prototyping and maintainable codebase.

I chose this combination because no other combination can offer the perfomance of the native parsing and processing engine combined with so rich and powerful Python interface.

\subsubsection{OpenSSL and pyOpenSSL}

``The OpenSSL Project is a collaborative effort to develop a robust, commercial-grade, full-featured, and Open Source toolkit implementing [\ldots] a full-strength general purpose cryptography library''\cite{openssl-homepage} This library is also written in C, has a unique Apache and BSD-like license, and is FIPS 140-2 compliant. PyOpenSSL provides a friendly object-oriented interface, which makes it possible to access all the functionality of OpenSSL I needed. It's also well-maintained, which makes installing it on modern OSes a breeze. I chose this duo, because it seemed the only solution capable of handling PEM files in all the ways I needed.

\subsubsection{XMLSec and PyXMLSec}

XMLSec is a C library based on Libxml2 and supports XML signature, encryption, and canonicalization.\cite{xmlsec-homepage} It's released under the MIT license, and is still maintained, so most Linux distributions provide it as an easily installable package. It uses libxml2 for XML processing and it can use several cryptography backends (OpenSSL, GnuTLS, Libgcrypt, NSS) for signature creation and encryption.

Python bindings were created for the Glasnost project financed by the French Department of Economy, Finance and Industry in 2003, but development seems to ceased around 2005. The bindings are still working, only one feature needed a patch sent to the mailing list of the project in 2010. The documentation consists of a dozen examples and an API reference generated from the source code, so the use of these bindings require quite a bit of experimentation.

There are few other projects trying to create XML signatures, with not much success, so I chose this one, because at least it worked, and with a bit of work, I managed to make it do what I wanted.
       %negyedik fejezet: a jelolt megoldasa
%\chapter{Results}

\section{Advantages of the new Python solution over using Apache CXF}

\subsection{Environment of measurement}

\subsection{Methodology of measurement}

\subsection{Analysis of the results}

\section{Observations}

\chapter{Summary}

\section{Results summary}

\section{Future development opportunities}

\subsection{Wider cryptographic backend support}

\subsection{Taking advantage of HTTP keep-alive}

\subsection{Implementing XML encryption}
       %Osszefoglalas: ertekeles, tovabbi munka

%\include{08_utozm}     %Utolso lapok: koszonetnyilvanitas, egyeb...

%
%Megjegyzés: célszerű használni BibTeX-et:

%(pl. egyszeru stilus:):
%\bibliography{mybib}
%\bibliographystyle{alpha}

%(pl. harvard stílus -- ez esetben a harvard.sty is betoltendo):
%\bibliography{mybib}
%\bibliographystyle{dcu}


\begin{thebibliography}{99}
\addcontentsline{toc}{chapter}{\bibname}

\bibitem{soa_modeling}
Bell, Michael (2008). Service-Oriented Modeling: Service Analysis, Design, and Architecture. Wiley & Sons. pp. 3. ISBN 978-0-470-14111-3.

\bibitem{devcom_soa_intro}
Stevens, Michael (April 16, 2002) -- Service-Oriented Architecture Introduction\\
\url{http://www.developer.com/services/article.php/1010451/Service-Oriented-Architecture-Introduction-Part-1.htm}

\bibitem{ibm_soa_impro}
Balzer, Yvonne (July 16, 2004) -- Improve your SOA project plans\\
\url{http://www.ibm.com/developerworks/webservices/library/ws-improvesoa/}

\bibitem{box_soap_history}
Box, Don (April 4, 2001) -- A Brief History of SOAP\\
\url{http://www.xml.com/pub/a/ws/2001/04/04/soap.html}

\bibitem{libxml2-homepage}
The XML C parser and toolkit of Gnome\\
\url{http://www.xmlsoft.org/}

\bibitem{lxml-homepage}
lxml - Processing XML and HTML with Python\\
\url{http://lxml.de/}

\bibitem{openssl-homepage}
OpenSSL: The Open Source toolkit for SSL/TLS\\
\url{http://openssl.org/}

\bibitem{xmlsec-homepage}
XML Security Library\\
\url{http://www.aleksey.com/xmlsec/}

\end{thebibliography}
       %Irodalomjegyzek

%%Fuggelek

\appendix

\chapter*{Appendix}
 \addcontentsline{toc}{chapter}{Appendix}
 \markboth{\uppercase{Appendix}}{\uppercase{Appendix}}
%\chaptermark{Függelék}

\setcounter{chapter}{1}     %A betu lesz

\blankpage
      %Fuggelek cimlap
%\section{Availability of relevant source code}

\subsection{SUDS}

Although I sent the patches enabling SUDS to digitally sign messages on May 21, 2011, as of December 2011, there's been no response from the maintainers. Because of this, the only way to obtain this code is my git repository hosted on GitHub.

\bigskip

\begin{center}
 \begin{minipage}[c]{0.7\linewidth}
  \begin{tabular}{rl}
   \textbf{Web access:} & \url{https://github.com/dnet/suds} \\
   \textbf{Git URL:} & \url{git://github.com/dnet/suds.git} \\
   \textbf{License:} & GNU LGPL version 3 (as it's part of SUDS)
  \end{tabular}
 \end{minipage}
 \hspace{5mm}
 \begin{minipage}[c]{3cm}
  \includegraphics[width=3cm]{images/qr/suds.png}
 \end{minipage}
\end{center}

\subsection{SudsSigner}

The component for creating WS\hyp{}Security digital signatures is implemented as a message plugin, and is to be considered a separate software. The source code can be downloaded from my git repository hosted on GitHub.

\bigskip

\begin{center}
 \begin{minipage}[c]{0.7\linewidth}
  \begin{tabular}{rl}
   \textbf{Web access:} & \small\url{https://github.com/dnet/SudsSigner} \\
   \textbf{Git URL:} & \small\url{git://github.com/dnet/SudsSigner.git} \\
   \textbf{License:} & MIT
  \end{tabular}
 \end{minipage}
 \hspace{5mm}
 \begin{minipage}[c]{3cm}
  \includegraphics[width=3cm]{images/qr/sudsign.png}
 \end{minipage}
\end{center}

\subsection{PyXMLSec}

The Python bindings for XMLSec were unmaintained since 2005, and the functionality SudsSigner needs can be only achieved in recent Python environments using a patch posted on the mailing list. I imported the Subversion repository of the project, applied the patch from the mailing list, and published this version of the code in my git repository hosted on GitHub.

\bigskip

\begin{center}
 \begin{minipage}[c]{0.7\linewidth}
  \begin{tabular}{rl}
   \textbf{Web access:} & \small\url{https://github.com/dnet/pyxmlsec} \\
   \textbf{Git URL:} & \small\url{git://github.com/dnet/pyxmlsec.git} \\
   \textbf{License:} & GNU GPL version 2
  \end{tabular}
 \end{minipage}
 \hspace{5mm}
 \begin{minipage}[c]{3cm}
  \includegraphics[width=3cm]{images/qr/pxsec.png}
 \end{minipage}
\end{center}
      %1. fuggelek
%\include{10_f2_me}      %2. fuggelek
%\include{10_f3_pr}      %3. fuggelek
%\include{10_f4_cd}      %4. fuggelek
%\include{10_f5_dk}      %5. fuggelek
%\include{10_f6_hl}      %6. fuggelek
%\include{10_f7_br}      %Biralat: későbbi kötéshez, opcionalis

%%abrak, tablazatok jegyzeke

\addcontentsline{toc}{chapter}{List of Figures}
\listoffigures

\addcontentsline{toc}{chapter}{List of Tables}
\listoftables
        %Esetleges abrak, tablazatok jegyzeke (lehet az elejen is, nem kotelezo)
%
\chapter*{Abbreviations}
 \addcontentsline{toc}{chapter}{Abbreviations}
 \markboth{\uppercase{Abbreviations}}{}

\begin{tabular}{p{20mm}p{120mm}}

  CORBA & Component Object Request Broker Architecture \\
  DCOM & Distributed Component Object Model \\
  HTTP & Hypertext Transfer Protocol \\
  JKS & Java Key Store \\
  PEM & Privacy Enhanced Mail \\
  PHP & PHP: Hypertext Preprocessor \\
  RPC & Remote Procedure Call \\
  SOA & Service-oriented Architecture \\
  SOAP & Simple Object Access Protocol \\
  SRP & Single Responsibility Principle \\
  TCP & Transport Control Protocol \\
  UDDI & Universal Description Discovery and Integration \\
  W3C & World Wide Web Consortium \\
  WSDL & Web Services Description Language \\
  XML & Extensible Markup Language \\

\end{tabular}
      %Roviditesek jegyzeke

\end{document}
