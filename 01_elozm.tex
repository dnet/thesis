\pagenumbering{roman}
%%%%%%%%%%%%%%%%%%%%%%%%%%%
% Diplomaterv-kiiras (ezt adjak, bele kell kötni a diplomába)
%%%%%%%%%%%%%%%%%%%%%%%%%%%
\begin{center}

\textbf{BUDAPEST UNIVERSITY OF TECHNOLOGY AND ECONOMICS}

\medskip

\includegraphics[width=8.79cm]{images/bme.pdf}

\medskip

\textbf{FACULTY OF ELECTRICAL ENGINEERING AND INFORMATICS\\SOFTWARE ENGINEERING}

 \vspace{2cm}
 \Large\textbf{\cim}

 \vspace{6mm}
 \textbf{\nev} \\
 \texttt{<vsza@vsza.hu>} \\ \strut \\

 \Large\textbf{THESIS STUDY}

\end{center}

\vfill

Consultant:

\begin{center}
\konzulens\\ \konzbeoszt

\vspace{96pt}

December 2011
\end{center}

% \vspace{6mm}
% \begin{tabular}{p{80mm}l}
% A záróvizsga tárgyai:   & Első tárgy \\
%                         & Második tárgy \\
%                         & Harmadik tárgy
%  \end{tabular}
%
%  \vspace{6mm}
%  \begin{tabular}{p{80mm}l}
%  A tervfeladat kiadásának napja:         &  \\
%  A tervfeladat beadásának határideje:    &
%  \end{tabular}
%
% \vfill
%
% \begin{center}
% \begin{tabular}{cc}
%  \makebox[7cm]{\emph{dr.\ Görgényi András}}    & \makebox[7cm]{\emph{dr.\ Péceli Gábor}} \\
%  \makebox[7cm]{adjunktus, diplomaterv felelős} & \makebox[7cm]{egyetemi tanár, tanszékvezető}
% \end{tabular}
% \end{center}
%
%  \vspace{6mm}
%  \begin{tabular}{p{80mm}l}
%  A tervet bevette:           & \\
%  A terv beadásának dátuma:   & \\
%  A terv bírálója:            &
%  \end{tabular}


 \thispagestyle{empty}
 \blankpage

\selectlanguage{magyar}
%%%%%%%%%%%%%%%%%%%%%%%%%%%
% Diplomaterv-kiiras melleklete (ezt is adjak, bele kell kötni a diplomába)
%%%%%%%%%%%%%%%%%%%%%%%%%%%
 \begin{textblock*}{\paperwidth}(0mm,0mm)
    \noindent\includegraphics[width=\paperwidth,height=\paperheight]{images/feladat-retouched.jpg}
	\end{textblock*}
	\mbox{}
 \blankpage

%%%%%%%%%%%%%%%%%%%%%%%%%%%
% Nyilatkozat
%%%%%%%%%%%%%%%%%%%%%%%%%%%
\def\abstractname{Nyilatkozat}
\begin{abstract}

\noindent
Alulírott \emph{Veres-Szentkirályi András}, szigorló hallgató kijelentem,
hogy ezt a diplomatervet meg nem engedett segítség nélkül, saját  magam
készítettem, csak a megadott forrásokat (szakirodalom, eszközök, stb.)
használtam fel. Minden olyan  részt, amelyet szó szerint, vagy azonos
értelemben, de átfogalmazva más forrásból átvettem, egyértelműen, a
forrás megadásával megjelöltem.

Hozzájárulok, hogy a jelen munkám alapadatait (szerző(k), cím, angol és magyar
nyelvű tartalmi kivonat, készítés éve, konzulens(ek) neve) a BME VIK nyilvánosan
hozzáférhető elektronikus formában, a munka teljes szövegét pedig az egyetem
belső hálózatán keresztül (vagy autentikált felhasználók számára) közzétegye.
Kijelentem, hogy a benyújtott munka és annak elektronikus verziója megegyezik.
Dékáni engedéllyel titkosított diplomatervek esetén a dolgozat szövege csak 3 év
eltelte után válik hozzáférhetővé.
\begin{flushright}
 \vspace*{1cm}
 \makebox[7cm]{\rule{6cm}{.4pt}}\\
 \makebox[7cm]{\emph{Veres-Szentkirályi András}}\\
 \makebox[7cm]{hallgató}
\end{flushright}
\end{abstract}

%%%%%%%%%%%%%%%%%%%%%%%%%%%
% Tartalomjegyzek
%%%%%%%%%%%%%%%%%%%%%%%%%%%
\selectlanguage{english}
\tableofcontents

%%%%%%%%%%%%%%%%%%%%%%%%%%%
% Kivonat
%%%%%%%%%%%%%%%%%%%%%%%%%%%
\def\abstractname{Kivonat}
\selectlanguage{magyar}
\begin{abstract}
\addcontentsline{toc}{chapter}{Kivonat}
% TODO
\end{abstract}


%%%%%%%%%%%%%%%%%%%%%%%%%%%
% Abstract
%%%%%%%%%%%%%%%%%%%%%%%%%%%
\selectlanguage{english}
\begin{abstract}
\addcontentsline{toc}{chapter}{Abstract}
``If I have seen further it is only by standing on the shoulders of giants'' -- the more than 300 years old message of Isaac Newton is a great parallel with the motivation behind service interoperation. As more advanced services were developed, one way of improving them was to interconnect them to combine their powers into more exciting products. Continuous advancement of technology created diverging platforms, and simultaneously provided standards allowing for efficient co-operation such as DCOM, CORBA and XML-RPC.

One of the solutions for this Babel-like chaos were web services, using SOAP to encode business messages in a standardized way, and reusing simple and mature transport mechanisms proven useful by the World Wide Web to carry them between the recipients. While the openness Internet offers endless opportunities, it also has its dangers, which caused additional standards to be developed, among others, for message authencity.

Although the open standards of SOAP, WSDL and others could have been the foundation of a platform-independent solution, not every environment used for software development supports it equally. Python, a truely community-driven project is one of them, providing little more than minimal support for advanced SOAP web services, making it a less favored selection for projects needing this capability, despite its unique treats.

In this thesis, the history and principles of Service-oriented Architecture are presented, then the scope is focused on web services, and further on to advanced ones. Then the Python environment is introduced, including the current solutions for implementing SOA solutions both on the service and consumer side. This part ends with a quick summary, that makes the reasons for improvements clear.

My selection for improvement, SUDS is presented next, looking at both its high-level view and its internals, showing the possible stubs awaiting improvement. The plans and implementation details of my enhancements are introduced right after, including a testbed to make sure, the new features fulfill the most important requirement: interoperation. In the end, the whole solution is evaluated using measurements of both timing and network traffic, concluding the thesis with my observations and ideas for future improvement.
\end{abstract}
