\chapter{Existing Python SOA solutions}

\section{About Python}

\subsection{Language}

According to \cite{python-faq}, ``Python is an interpreted, interactive, object-oriented programming language''. What's missing from this self-description are those things that make the language unique. The first thing most people recognize while reading a Python source code, is the use of indentation for structure markup. This feature might be odd and strict for first sight, but it makes code written by other people highly readable and reusable. The subsequent clean feeling of the code is strengthened even more by the availability of functional constructs, which give the right tools for most purposes into the hands of the developer. The essence of the language can be reduced to a single phrase, which can be interpreted in both positive and negative ways: ``executable pseudocode'' -- code snippets are explicit enough for most people (even those without Python knowledge) to understand.

\subsection{Runtime}

The first and most widely-used interpreter is called CPython, and it's the reference implementation of the language runtime. It compiles source code (\verb|.py| files) into bytecode (\verb|.pyc| files) for interpretation. It also provides an interactive shell, which can be used for experimentation or debug purposes. Because of this, the UNIX program loader can use it as a standard interpreter, so Python scripts prefixed with an appropriate shebang can be run directly. As the name suggests, the implementation is written in mostly C/C++, which causes built-in functions to perform well.

There are separate projects that bridge the Python world with other solutions -- IronPython and Jython compiles Python code into .NET and Java bytecode, respectively, and Nokia ported Python to its S60 (Symbian) platform. This way, Python can interoperate with existing frameworks and libraries at a lower level, if needed. Another approach is outlined in the next subsection.

\subsection{Libraries}

Python comes with ``batteries included'' -- libraries are available for most purposes a developer might need, such as file manipulation, network connectivity, parsing and serializing from and to a variety of formats. Libraries can be either written in Python -- in which case, they are as portable as any other Python code between runtimes -- or using the C/C++ API. Thin, sometimes automatically generated libraries, that only wrap certain native libraries are called bindings -- there's even a dialect of Python called Cython that allows calling of C/C++ functions, and produces native code. Because of these features, although interpreted languages are usually suffer from poor performance, well-designed Python applications perform only high-level orchestration in the interpreted engine, and delegate computationally intensive tasks to native code. This design motivates developers to avoid premature optimization, while allowing fast prototyping and outstanding performance using the same foundations.

\section{Python SOA solutions}

\subsection{Introduction}

The community around the Python ecosystem is one of those closest to the ``free software culture'' envisioned by Richard M. Stallman and Eric S. Raymond. This results in libraries being written mostly out of curiosity and immediate need -- a good combination for a good base system, not so good for SOAP. Many other ways of remote method invocation are solved in Python libraries, but SOAP has maintained a low level of maturity. The basic invocation examples usually work, but the level of development clearly shows the needs of the developer.

This problem is partly caused by the network effect: if everybody uses .NET and Java for web service interoperation, if a platform needs to be chosen for a solution to access them, it usually seems logical for most people to choose one of the two heavyweight products. The other factor is the mindset of Python developers -- they usually like to build systems out of small, autonomous entities, interconnected by simple and trivial protocols, and SOAP is not the first thing that comes to mind with these features, despite the fact that it can be used wisely.

Of course, there are developers both working on and using SOA with Python, there's even a public mailing list dedicated for the purpose, archives and subscription are available at \url{http://mail.python.org/mailman/listinfo/soap}.

\subsection{SOAPy}

As \cite{so-206154} wrote, it was the best SOAP client in the Python ecosystem, but the project is abandoned. As no one maintains the codebase -- its homepage was last modified in 2001 -- it became incompatible with later Python releases, which makes it hard to use in modern environments. The Debian project doesn't even maintain a package, so the only way to install it is to download the tar.gz file (uploaded in April 27, 2001) and extract it manually.

I found its internal structure very simple -- the library consists of two Python source files, both under 500 lines of length. By looking at the import section, it was obvious that it used libraries and functions that are way obsolete now. Still, some of them are kept for the sake of backwards compatibility, but the PyXML package it used for XML processing is also no longer maintained, and had been removed from most major GNU/Linux distributions. The documentation -- including the examples -- suggested that SOAPy offered client functionality only, and I didn't find any contradicting evidence in the source code.

\subsection{Zolera SOAP Infrastructure}
\label{ZSI}

ZSI is the other ``old boy'' among Python SOAP libraries. As \cite{pywebsvcs-talk} states, it was last fully released in 2007, but unlike SOAPy, it can still be used with recent Python environments. It offers two ways of operation: for simple services it can construct the SOAP messages without a schema (Binding class), and for complex services a proxy (ServiceProxy class) can be used to serialize arguments. It supports code generation from WSDL (\verb|wsdl2py|), and ``can also be used to build applications using SOAP Messages with Attachments'' \cite{zsi-doc}.

Ralf Schmitt wrote in \cite{zsi-velocity} that ZSI is neither easy to set up and use, nor fast. I tried it anyway, and it was easy to install, since Debian still maintains a package. The first ServiceProxy example, which I took straight from the ZSI documentation failed, but I figured out that they changed the structure, so I managed to run a test. It worked pretty well as a client, but it lacked any advanced debugging features -- for example, the list of methods could only be determined by listing the methods of the service proxy. Also, code (re)generation is necessary for complex data types, and is far from being trivial.

\subsection{soaplib / rpclib}

Soaplib focused on server-side SOAP implementation, using Python decorators, and provided WSGI-compatible services, so deployment was possible with both standalone processes or any WSGI-compatible web server (such as Apache \verb|mod_wsgi|). \cite{so-206154} wrote that ``creating clients is a little bit more challenging'', so I looked through the documentation and found that according to \cite{soaplib2-changelog}, the developers did ``the right thing'' and shifted their entire focus on server implementation by dropping client functionality in favor of SUDS (see section \ref{suds}) at version 0.9.

The project was later renamed to rpclib, and widened its scope -- the old library only receives bug fixes, as the main developer focuses on the new one. I tried using it, and found it pleasant to use. Despite Python being a dynamically typed language, enabling code to be accessible via SOAP had not ``littered'' the code, and it offered automatic WSDL generation.

\subsection{SUDS}
\label{suds}

SUDS is a relatively new SOAP client library, compatible with Python 2.4 and newer releases. Its operation is like the proxy feature of ZSI, but it doesn't require any code generation. Complex classes can be assembled using the factory pattern, and while it might seem that parsing WSDL and generating class hierarchy on-the-fly is slow, the built-in caching provides quite a performance. It supported several methods of authentication, including HTTP basic and digest, and also NTLM, which is necessary to consume Microsoft SharePoint web services. According to the general opinion of related forums and mailing lists (including \cite{so-206154}), SUDS is the preferred Python way of creating SOAP clients, and the library doesn't depend on obsolete components.

SUDS was released in a regular manner till 2010, and is available as a package in major Linux distributions. This way installing the library was not a big issue, and the documentation \cite{suds-doc} covers all common use-cases. First I tried it with a basic invocation, and it worked as expected. Special methods were overridden in a way that using the \verb|print| command on SUDS object rendered a nicely formatted, human readable printout, which makes debugging and experimentation much easier.

\subsection{sec-wall}

Although \cite{sec-wall-homepage} describing sec-wall as ``a security proxy that comes with tons of interesting features, very good documentation and an exceptionally friendly community'' might sound like the usual shameless self-promotion, this relatively new tool (1.0 released in April 2011) is a real gem. It acts as a proxy, thus enables the transformation of any SOAP backend into an advanced web service. Although written in Python could have meant poor performance, it takes the issue seriously and makes use of libraries that provide a native event-driven architecture.

I found it during the field-work, and I worked together with its author, Dariusz Suchojad to improve it -- one of the results were complete and correct UsernameToken support (both plain and digest), and an experimental digital signature implementation. Beside the performance and reusability, the quality of the software is also surprisingly great; it's built around the Python Spring Framework, making good use of the dependency injection feature, and its tests provide 100\% code coverage.

\subsection{Common problems}

While inspecting libraries offering both service and consumer functionality, unfortunately they provided no or little support for advanced web services. SUDS offered UsernameToken, but didn't work in any mode, soaplib/rpclib and ZSI didn't even mention the possibility of such solutions -- although ZSI had some unused code implementing XML canonicalization. SOAPy barely even implemented SOAP -- besides it's unusable in modern environments. Although sec-wall solves the situation by providing proxy support, the problem of the client side remains -- and that's exactly why I decided to take a close look into SUDS.
