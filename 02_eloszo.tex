%Eloszo

\chapter*{Introduction}

 \addcontentsline{toc}{chapter}{Introduction}
 \markboth{\uppercase{Introduction}}{\uppercase{Introduction}}
 \pagenumbering{arabic}

% TODO background in five paragraphs

% sum of chapters in two paragraphs
The first chapter introduces the Service-oriented Architecture and web services, including their brief history and principles. Using these as a basis, the scope is first focused on advanced web services, and then on the security of web services. The second chapter sheds some light on the other half of the thesis title, Python, and presents the available SOA solutions compatible with the platform, along with their advantages and disadvantages. The chapter ends with a quick summary of the Python SOA landscape, picking SUDS as a suitable base of improvement.

The third chapter goes into detail about SUDS, starting with its interface presented to the developer, then focusing on the security-related parts of its internals, ending in its interestring stubs awaiting improvement. The fourth chapter starts from the deficiencies of SUDS, and introduces the component I designed to enable advanced web service consumptions. In the second half of the chapter, a testbed is shown, which I developed to make sure, that the result of the improvement is able to interoperate with services correctly. The fifth chapter demonstrates the measurements I did to evaluate the quality of my solution, and the sixth closes the thesis study by summarizing the development process and its results, offering future improvement ideas to address the remaining issues.
