%Eloszo

\chapter*{Introduction}

 \addcontentsline{toc}{chapter}{Introduction}
 \markboth{\uppercase{Introduction}}{\uppercase{Introduction}}
 \pagenumbering{arabic}

% background in five paragraphs
Looking at the history of ITC systems, interoperability was not a big issue at the beginning -- simple systems can communicate using primitive methods. As time went by, systems evolved, and innovation lead to such diversity that high level interconnection of systems became a major headache of system integrators. Naturally, the market reacted and came up with various solutions, well distributed along the scale of bloatedness, including DCOM, CORBA and XML-RPC.

These competing solutions were and are more or less usable within their scopes, limited by platform-dependence, but more importantly the inability to operate over the Internet -- either because of the incompatibility with appliances at network borders, or because of security issues. The World Wide Web brought simple open protocols, and mature solutions to pass its network traffic through network borders, making it an ideal choice as the transport layer for the next generation of interoperability platforms.

SOAP was created as the encoding method of request, reply and fault messages exchanged between services and consumers over the transport layer, but it didn't solve all problems in itself. For instance, trusting a network out of control of both parties required additional standardized ways of ensuring the confidentiality and integrity of the transmitted message, as well as authenticating the consumer and/or the service. In case of this problem, WS\hyp{}Security was born as a solution, providing a simple and open method of ensuring the necessary level of security for SOAP messages relayed over untrusted networks.

Python was one of the few languages that -- despite its roots -- could emerge from the academic circles, and found its way to developers, hackers, and system administrators alike. The rich set of libraries and sane design made it a perfect choice for high-level implementation, allowing a smooth transition from ideas through prototypes to solutions ready for deployment -- the same feature that caused many people writing it off as a ``scripting language''. As was Java before the millennium, Python is currently considered by many as the language of the Internet (or at least one of them), which means, advanced SOAP support is a must-have in order for Python to be accepted as the foundation of a wider range of systems.

One of the negative side effects of the community-driven development of the Python ecosystem is that features needed by less people get a smaller fraction of developer attention -- and this was exactly the fate of advanced Python SOA implementations. Although SOAP libraries did exist, their support was limited to the level the developer(s) required, resulting in many organizations choosing other platforms solely based on their advanced SOAP support -- creating a situation I refused to accept.

% sum of chapters in two paragraphs
The first chapter introduces the Service-oriented Architecture and web services, including their brief history and principles. Using these as a basis, the scope is first focused on advanced web services, and then on the security of web services. The second chapter sheds some light on the other half of the thesis title, Python, and presents the available SOA solutions compatible with the platform, along with their advantages and disadvantages. The chapter ends with a quick summary of the Python SOA landscape, picking SUDS as a suitable base of improvement.

The third chapter goes into detail about SUDS, starting with its interface presented to the developer, then focusing on the security-related parts of its internals, ending in its interesting stubs awaiting improvement. The fourth chapter starts from the deficiencies of SUDS, and introduces the component I designed to enable advanced web service consumptions. In the second half of the chapter, a testbed is shown, which I developed to make sure that the result of the improvement is able to interoperate with services correctly. The fifth chapter demonstrates the measurements I did to evaluate the quality of my solution, and the sixth closes the thesis study by summarizing the development process and its results, offering future improvement ideas to address the remaining issues.
